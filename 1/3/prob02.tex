\documentclass{article}
\usepackage{amsmath}
\usepackage{amssymb}
\usepackage{textcomp}
\usepackage{gensymb}
\begin{document}

\textbf{1.3.2} Calculate the work required to move a mass \textit{m} against a force field $\mathbf{F} = y\mathbf{u_x} +
	x\mathbf{u_y}$ along the path \textit{abc} and along the path \textit{adc}. Is this field conservative?
\textit{Note: Diagram not shown.}

\vspace{24pt}

By inspection of the diagram, the labeled points are $a = (0, 0)$, $b = (0, 5)$, $c = (5, 5)$, and $d = (0, 5)$. To
find the work required to move along the two specified paths, we need to compute the path integrals of the force field.

\begin{equation*}
	\begin{split}
		W_{abc} & = \int_a^b \mathbf{F} \bullet \mathbf{dl} + \int_b^c \mathbf{F} \bullet \mathbf{dl} \\
		& = \int_a^b (y\mathbf{u_x} + x\mathbf{u_y}) \bullet (\text{d}x\mathbf{u_x} + \text{d}y\mathbf{u_y}) + \int_b^c (y\mathbf{u_x} + x\mathbf{u_y}) \bullet (\text{d}x\mathbf{u_x} + \text{d}y\mathbf{u_y}) \\
		& = \int_a^b y\text{d}x + \int_a^b x\text{d}y + \int_b^c y\text{d}x + \int_b^c x\text{d}y \\
		& = \int_{x=0}^0 y\text{d}x + \int_{y=0}^5 x\text{d}y + \int_{x=0}^5 y\text{d}x + \int_{y=5}^5 x\text{d}y \\
		& = \int_{y=0}^5 0\text{d}y + \int_{x=0}^5 5\text{d}x \\
		& = 5x \big\rvert_{x=0}^5 \\
		& = 25
	\end{split}
\end{equation*}

\begin{equation*}
	\begin{split}
		W_{adc} & = \int_a^d \mathbf{F} \bullet \mathbf{dl} + \int_d^c \mathbf{F} \bullet \mathbf{dl} \\
		& = \int_a^d (y\mathbf{u_x} + x\mathbf{u_y}) \bullet (\text{d}x\mathbf{u_x} + \text{d}y\mathbf{u_y}) + \int_d^c (y\mathbf{u_x} + x\mathbf{u_y}) \bullet (\text{d}x\mathbf{u_x} + \text{d}y\mathbf{u_y}) \\
		& = \int_a^d y\text{d}x + \int_a^d x\text{d}y + \int_d^c y\text{d}x + \int_d^c x\text{d}y \\
		& = \int_{x=0}^0 y\text{d}x + \int_{y=0}^5 x\text{d}y + \int_{x=0}^5 y\text{d}x + \int_{y=5}^5 x\text{d}y \\
		& = \int_{y=0}^5 0\text{d}y + \int_{x=0}^5 5\text{d}x \\
		& = 5x \big\rvert_{x=0}^5 \\
		& = 25
	\end{split}
\end{equation*}

The fact that both of the given paths have equal values for their path integrals is not sufficient to prove that
$\mathbf{F}$ is conservative. It is possible that there are other paths between \textit{a} and \textit{c} that produce
different values, and we cannot check all of them because there are infinitely many. However, it is sufficient to show
that there is a scalar function $f$ that satisfies $\mathbf{F} = \nabla f$, by the definition of a conservative field.
That function is $f = xy$.

\begin{equation*}
	\begin{split}
		\mathbf{F} & = \nabla f \\
		y\mathbf{u_x} + x\mathbf{u_y} & = \frac{\partial f}{\partial x}\mathbf{u_x} + \frac{\partial f}{\partial y}\mathbf{u_y} \\
		y\mathbf{u_x} + x\mathbf{u_y} & = \frac{\partial}{\partial x}(xy)\mathbf{u_x} + \frac{\partial}{\partial y}(xy)\mathbf{u_y} \\
		y\mathbf{u_x} + x\mathbf{u_y} & = y\mathbf{u_x} + x\mathbf{u_y}
	\end{split}
\end{equation*}

Therefore, $\mathbf{F}$ is conservative.

\end{document}