\documentclass{article}
\usepackage{amsmath}
\usepackage{amssymb}
\usepackage{textcomp}
\usepackage{gensymb}
\usepackage{geometry}
\usepackage{mathrsfs}
\begin{document}

\textbf{1.4.15} By direct differentiation, show that $\nabla^2 (\frac{1}{r}) = 0$ at all points where $r \neq 0$ where
$r = \sqrt{{(x-x')}^2 + {(y-y')}^2 + {(z-z')}^2}$.

\vspace{24pt}

From \textbf{1.4.3}, we know that

\begin{equation*}
	\nabla \left(\frac{1}{r}\right) = -{(r^2)}^{-\frac{3}{2}}(x - x')\mathbf{u_x} - {(r^2)}^{-\frac{3}{2}}(y - y')\mathbf{u_y} - {(r^2)}^{-\frac{3}{2}}(z - z')\mathbf{u_z}
\end{equation*}

which we can use in the following way.

\begin{equation*}
	\begin{split}
		\nabla^2 \left(\frac{1}{r}\right) & = \nabla \bullet \nabla \frac{1}{r} \\
		& = \nabla \bullet \left[
		-{(r^2)}^{-\frac{3}{2}}(x - x')\mathbf{u_x} - {(r^2)}^{-\frac{3}{2}}(y - y')\mathbf{u_y} - {(r^2)}^{-\frac{3}{2}}(z - z')\mathbf{u_z}
		\right] \\
		& = \frac{\partial}{\partial x}\left[-{(r^2)}^{-\frac{3}{2}}(x - x')\right]
		+ \frac{\partial}{\partial y}\left[-{(r^2)}^{-\frac{3}{2}}(y - y')\right]
		+ \frac{\partial}{\partial z}\left[-{(r^2)}^{-\frac{3}{2}}(z - z')\right] \\
		& = \frac{3}{2}{(r^2)}^{-\frac{5}{2}}(2x - 2x')(x - x') - {(r^2)}^{-\frac{3}{2}} \\
		& \quad + \frac{3}{2}{(r^2)}^{-\frac{5}{2}}(2y - 2y')(y - y') - {(r^2)}^{-\frac{3}{2}} \\
		& \quad + \frac{3}{2}{(r^2)}^{-\frac{5}{2}}(2z - 2z')(z - z') - {(r^2)}^{-\frac{3}{2}} \\
		& = 3r^{-5}{(x - x')}^2 + 3r^{-5}{(y - y')}^2 + 3r^{-5}{(z - z')}^2 - 3r^{-3} \\
		& = 3r^{-5}\left[{(x - x')}^2 + {(y - y')}^2 + {(z - z')}^2\right] - 3r^{-3} \\
		& = 3r^{-5} \cdot r^2 - 3r^{-3} \\
		& = 3r^{-3} - 3r^{-3} \\
		& = 0
	\end{split}
\end{equation*}

Therefore, the Laplacian of the inverse of $r$ is equal to $0$, except when $r \neq 0$, where it is undefined.

\end{document}