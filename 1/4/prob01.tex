\documentclass{article}
\usepackage{amsmath}
\usepackage{amssymb}
\usepackage{textcomp}
\usepackage{gensymb}
\usepackage{mathrsfs}
\begin{document}

\textbf{1.4.1} A hill can be modeled with the equation $H = 10 - x^2 - 3y^2$ where $H$ is the elevation of the hill.
Find the path that a frictionless ball would take to experience the greatest change of elevation in the least change of
horizontal position. Assume that the motion of the ball is unconstrained.

\vspace{24pt}

The path that the ball would take is equivalent to the gradient of $H$.

\begin{equation*}
	\begin{split}
		\nabla H & = \frac{\partial H}{\partial x}\mathbf{u_x} + \frac{\partial H}{\partial y}\mathbf{u_y} \\
		& = \frac{\partial}{\partial x}(10 - x^2 - 3y^2)\mathbf{u_x} + \frac{\partial}{\partial y}(10 - x^2 - 3y^2)\mathbf{u_y} \\
		& = -2x\mathbf{u_x} - 6y\mathbf{u_y}
	\end{split}
\end{equation*}

\end{document}