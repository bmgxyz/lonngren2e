\documentclass{article}
\usepackage{amsmath}
\usepackage{amssymb}
\usepackage{textcomp}
\usepackage{gensymb}
\begin{document}

\textbf{1.2.14} Find the area of the parallelogram using vector notation. Compare your result with that found
graphically.\@ \textit{Note: Diagram not shown.}

\vspace{24pt}

The magnitude of the vector product of two vectors is equal to the area of the parallelogram formed by those vectors.
Let the two vectors $\mathbf{A} = \begin{pmatrix} 2 & 4 & 0 \end{pmatrix}$ and $\mathbf{B} = \begin{pmatrix} 6 & 0 & 0
	\end{pmatrix}$ be the two adjacent sides of the parallelogram that start at the origin. Note that both vectors have a
$\mathbf{u_z}$ component that is equal to zero because the vector product is defined in three dimensions, and
$\mathbf{A}$ and $\mathbf{B}$ lie in the $xy$ plane.

\begin{equation*}
	\begin{split}
		||\mathbf{A} \times \mathbf{B}|| & = ||(2\mathbf{u_x} + 4\mathbf{u_y} + 0\mathbf{u_z}) \times (6\mathbf{u_x} + 0\mathbf{u_y} + 0\mathbf{u_z})|| \\
		& = ||0\mathbf{u_x} + 0\mathbf{u_y} - 24\mathbf{u_z}|| \\
		& = \sqrt{0^2 + 0^2 + {(-24)}^2} \\
		& = 24
	\end{split}
\end{equation*}

This value matches the result obtained when counting the area graphically. The parallelogram may be split into a central
square with side length $4$ and two triangles that together form a rectangle of $2$ by $4$ units. This gives an area of
$4 \cdot 4 + 2 \cdot 4 = 24$.

\end{document}