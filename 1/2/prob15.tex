\documentclass{article}
\usepackage{amsmath}
\usepackage{amssymb}
\usepackage{textcomp}
\usepackage{gensymb}
\begin{document}

\textbf{1.2.15} Show that we can use the vector definitions $\mathbf{A} \bullet \mathbf{B} = 0$ and $\mathbf{A} \times
	\mathbf{B} = \mathbf{0}$ to express that two vectors are perpendicular and parallel to each other respectively.

\vspace{24pt}

The definitions of the scalar and vector products are $\mathbf{A} \bullet \mathbf{B} = AB\cos{\theta}$ and $\mathbf{A}
	\times \mathbf{B} = AB\sin{\theta}\mathbf{u_{\mathbf{A} \times \mathbf{B}}}$, where $\theta$ is the angle between
$\mathbf{A}$ and $\mathbf{B}$. Assume that $\mathbf{A}$ and $\mathbf{B}$ are both nonzero. If either is the zero
vector, then the vectors are neither perpendicular nor parallel.

When the scalar product is equal to zero, the value of $\theta$ must be $\pm\frac{\pi}{2}$ because $\cos{\theta} = 0$ for
those values. Vectors are perpendicular when the angle between them is $\pm\frac{\pi}{2}$, so the vectors must be
perpendicular when their scalar product is zero.

By a similar argument, when the vector product is equal to zero, the value of $\theta$ must be $0$ or $\pi$ because
$\sin{\theta} = 0$ for those values. Vectors are parallel when the angle between them is $0$ or $\pi$, so the vectors
must be parallel when their vector product is equal to zero.

\end{document}