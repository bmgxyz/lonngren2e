\documentclass{article}
\usepackage{amsmath}
\usepackage{textcomp, gensymb}
\begin{document}

\textbf{1.2.3} Find the scalar product of the two vectors defined by $\mathbf{A} = 3\mathbf{u_x} + 4\mathbf{u_y} +
	5\mathbf{u_z}$ and $\mathbf{B} = -5\mathbf{u_x} + 4\mathbf{u_y} - 3\mathbf{u_z}$. Determine the angle between these
two vectors. Check your answer using MATLAB\@.

\vspace{24pt}

The scalar product of two vectors is equal to the sum of the products of their components. It isn't necessary to fully
expand the product because the cross terms cancel (e.g., $\mathbf{u_x} \cdot \mathbf{u_y} = 0$).

\begin{equation*}
	\begin{split}
		\mathbf{A} \bullet \mathbf{B} & = (3\mathbf{u_x} + 4\mathbf{u_y} + 5\mathbf{u_z})(-5\mathbf{u_x} + 4\mathbf{u_y} -
		3\mathbf{u_z}) \\
		& = (3\mathbf{u_x} \cdot -5\mathbf{u_x}) + (4\mathbf{u_y} \cdot 4\mathbf{u_y}) + (5\mathbf{u_z} \cdot
		-3\mathbf{u_z}) \\
		& = -15 + 16 - 15 \\
		& = -14
	\end{split}
\end{equation*}

The scalar product of two vectors is also equal to the product of their magnitudes and the cosine of the angle between
them. Rearranging for the angle, we have

\begin{equation*}
	\begin{split}
		\mathbf{A} \bullet \mathbf{B} & = AB\cos{\theta} \\
		-14 & = ||\mathbf{A}||||\mathbf{B}||\cos{\theta} \\
		-14 & = \sqrt{3^2 + 4^2 + 5^2}\sqrt{{(-5)}^2 + 4^2 + {(-3)}^2}\cos{\theta} \\
		-14 & = 50\cos{\theta} \\
		\frac{-14}{50} & = \cos{\theta} \\
		\theta & = \cos^{-1}{\frac{-7}{25}} \approx 106\degree.
	\end{split}
\end{equation*}

\end{document}